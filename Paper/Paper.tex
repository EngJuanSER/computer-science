\documentclass[conference]{IEEEtran}
\IEEEoverridecommandlockouts 
\usepackage{cite}
\usepackage[cmex10]{amsmath,amssymb}
\usepackage{algorithmic}
\usepackage{graphicx}
\usepackage{textcomp}
\usepackage{xcolor}
\usepackage{hyperref}
\usepackage{enumitem}
\usepackage{tikz}

\markboth{MacraScript: DSL para Patrones de Macramé}{}

\begin{document}

\title{MacraScript: Un Lenguaje de Programación Específico de Dominio para la Creación de Patrones de Macramé}

\author{\IEEEauthorblockN{Juan Manuel Serrano Rodriguez - 20211020091, Javier Alejandro Sánchez Salamanca - 20201020167}
\IEEEauthorblockA{Depto. de Ingeniería de Sistemas \\
Universidad Francisco José de Caldas\\
Bogotá, Colombia \\
\{jumserranor, jasanchezs\}@udistrital.edu.co}
\thanks{Proyecto desarrollado para la asignatura Ciencias de la Computación 3}}

\maketitle

\begin{abstract}
This work presents MacraScript, a domain-specific programming language (DSL) developed for the creation and visualization of macramé patterns. Macramé, a craft technique based on decorative knots, faces challenges in planning and digital design that can be addressed through a programmatic approach. MacraScript offers a declarative syntax to specify threads, colors, and knot sequences, with the goal of generating both a visual preview of the final result and detailed execution guides. This paper describes the motivation, problem identification, and conceptual proposal of the language. MacraScript seeks to bridge the gap between programming and traditional crafts, facilitating the creation of complex macramé patterns through computational tools.
\end{abstract}

\begin{IEEEkeywords}
domain-specific languages, macramé, textile patterns, graphics generation, interpreters, digital crafts
\end{IEEEkeywords}

\section{Introducción}
El macramé es una forma antigua de arte textil basada en la creación de patrones decorativos mediante secuencias organizadas de nudos. Esta técnica artesanal, que data de miles de años, sigue siendo popular en la actualidad para crear desde accesorios personales hasta elementos decorativos elaborados~\cite{karner2005}. Sin embargo, el diseño y planificación de patrones de macramé complejos presenta desafíos significativos: requiere una meticulosa atención al detalle, es propenso a errores de conteo y carece de herramientas digitales especializadas para su diseño y visualización.

Las herramientas actuales incluyen el dibujo manual en papel cuadriculado, software genérico de diseño gráfico o aplicaciones de arte en píxeles (pixel art), ninguna optimizada para las necesidades específicas de esta técnica. En particular, los patrones de tipo "alpha" (similares al pixel art) y los patrones geométricos tradicionales requieren una planificación cuidadosa para lograr el efecto visual deseado.

Este trabajo presenta MacraScript, un lenguaje de programación específico de dominio (DSL) desarrollado para abordar estas limitaciones. MacraScript permitiría a los diseñadores definir patrones de macramé mediante código, especificando parámetros como el número de hilos, colores y secuencias de nudos. La propuesta contempla dos salidas visuales principales: una vista previa del diseño finalizado y una guía paso a paso que muestre la secuencia de nudos necesaria.

Al permitir un enfoque programático, MacraScript ofrecería ventajas como la capacidad de reutilización de patrones, modificación paramétrica de diseños existentes y validación automática antes de su implementación física. Este paper presenta la conceptualización inicial de la propuesta.

\section{Métodos y Materiales}
El desarrollo de MacraScript se basa en principios fundamentales de diseño de lenguajes específicos de dominio y generación de gráficos por computadora, aplicados al contexto artesanal del macramé.

La investigación sigue una metodología de diseño iterativo, comenzando con el análisis de los requerimientos específicos de los diseñadores de macramé y las deficiencias de las herramientas existentes. Para abordar estas necesidades, proponemos un DSL que permita expresar patrones de macramé de forma programática pero accesible.

Los DSL son lenguajes de programación optimizados para aplicaciones particulares, ofreciendo abstracciones y notaciones adaptadas al dominio~\cite{fowler2010}. MacraScript se plantea como un DSL externo con sintaxis declarativa, permitiendo especificar hilos, colores y secuencias de nudos de manera natural para el dominio~\cite{voelter2013}.

Para la representación computacional, la propuesta contempla un modelo de datos capaz de representar tanto la disposición inicial de los hilos y sus propiedades como la taxonomía de nudos y sus relaciones espaciales~\cite{hashemi2017}. Las salidas visuales aprovecharán técnicas establecidas de gráficos por computadora, utilizando rasterización para previsualizaciones y vectores para guías detalladas~\cite{hudson2018}.

El problema central que abordamos es la ausencia de herramientas digitales especializadas para el diseño de macramé. Las soluciones actuales (dibujo manual, software genérico de diseño, aplicaciones de pixel art) presentan limitaciones significativas: propensión a errores, dificultad para visualizar resultados finales, complejidad para realizar modificaciones y ausencia de estándares para compartir diseños~\cite{owen1997}.

\section{Resultados}
Como resultado principal de esta investigación, presentamos el diseño conceptual de MacraScript, un DSL que busca facilitar la creación, visualización y compartición de patrones de macramé.

El lenguaje propuesto introduce un modelo declarativo con una sintaxis intuitiva para diseñadores familiarizados con la terminología del macramé. MacraScript se centra en el concepto de PATTERN como unidad fundamental, permitiendo definir propiedades específicas como tipo de patrón, hilos, colores y secuencias de nudos~\cite{voelter2013}.

A continuación se muestra un ejemplo simplificado de la sintaxis para un patrón tipo "alpha", similar al pixel art:

\begin{verbatim}
PATTERN MyFirstAlpha {
    TYPE: ALPHA;
    THREADS: 12;
    BACKGROUND_COLOR: "white";
    DEFINE_COLOR: C1 = "black";
    DEFINE_COLOR: C2 = "green_light";
    
    ROW 1: [C1, C1, C2, C2, C1, C1];
    ROW 2: [C1, C2, C1, C1, C2, C1];
    ROW 3: [C2, C1, C1, C1, C2, C2];
}
\end{verbatim}

Para patrones con nudos estructurales tridimensionales, la sintaxis propuesta permite especificar secuencias de nudos específicas:

\begin{verbatim}
PATTERN ChevronBracelet {
    TYPE: NORMAL;
    THREADS: 8;
    DEFINE_COLOR: A = "red";
    DEFINE_COLOR: B = "blue";
    SETUP_THREADS: [A,A,B,B,B,B,A,A];

    ROW { 
        KNOTS: [(1,2,"fk"), 
                (3,4,"fk"), 
                (5,6,"fk")]; 
    }
    REPEAT ROW 5 TIMES;
}
\end{verbatim}

La evaluación preliminar del diseño del lenguaje muestra potencial para reducir significativamente la curva de aprendizaje y los errores comunes en el diseño de patrones de macramé. Las pruebas de concepto indican que la sintaxis es suficientemente expresiva para representar tanto patrones simples como complejos, y la estructura del lenguaje facilita la reutilización y adaptación de componentes.

\section{Desafíos y Trabajo Futuro}
Aunque el diseño conceptual de MacraScript muestra un potencial prometedor, identificamos desafíos técnicos que requieren trabajo futuro. Particularmente importante es la formalización completa de la gramática, que debe ser lo suficientemente expresiva para cubrir la diversidad de técnicas y patrones del macramé tradicional~\cite{mernik2005}. 

La representación visual es otro aspecto crítico: necesitamos desarrollar simbologías claras y consistentes para los diferentes tipos de nudos, manteniendo un balance entre precisión técnica y comprensibilidad~\cite{hashemi2017}. La simulación visual precisa del comportamiento de los hilos y nudos también presenta complejidades computacionales significativas~\cite{hudson2018}.

El trabajo futuro se centrará en completar el diseño formal de la sintaxis y semántica, desarrollar un intérprete prototipo, y evaluar su usabilidad con artesanos reales. También contemplamos el desarrollo de una biblioteca de patrones comunes y la posibilidad de integrar técnicas de inteligencia artificial para sugerir modificaciones y optimizaciones en los diseños.

\section{Conclusión}
MacraScript representa un puente innovador entre la programación computacional y el arte ancestral del macramé. El desarrollo de este DSL promete transformar la manera en que los diseñadores crean, modifican y comparten patrones de macramé, democratizando el acceso a esta forma de expresión artística~\cite{rochford2019}.

La propuesta presentada demuestra cómo los principios de la ciencia de la computación pueden aplicarse con éxito a dominios creativos tradicionalmente manuales, potenciando la expresión artística a través de abstracciones computacionales adecuadas. Al formalizar el lenguaje visual del macramé en una sintaxis programática, no solo facilitamos la creación de diseños más complejos y precisos, sino que abrimos posibilidades para la emergencia de nuevas formas de creatividad asistida por computadora~\cite{owen1997}.

\begin{thebibliography}{00}
\bibitem{fowler2010} Fowler, M. (2010). Domain-Specific Languages. Addison-Wesley Professional.
\bibitem{mernik2005} Mernik, M., Heering, J., \& Sloane, A. M. (2005). When and how to develop domain-specific languages. ACM Computing Surveys (CSUR), 37(4), 316-344.
\bibitem{voelter2013} Voelter, M., Benz, S., Dietrich, C., Engelmann, B., Helander, M., Kats, L. C., \& Wachsmuth, G. (2013). DSL Engineering: Designing, Implementing and Using Domain-Specific Languages. CreateSpace Independent Publishing Platform.
\bibitem{owen1997} Owen, C. L. (1997). Design research: Building the knowledge base. Design Studies, 19(1), 9-20.
\bibitem{rochford2019} Rochford, K. (2019). Textile Patterns as Code. Textile: Journal of Cloth and Culture, 17(4), 368-380.
\bibitem{karner2005} Karner, C. (2005). Macramé: The craft of knotting. Schiffer Publishing.
\bibitem{hudson2018} Hudson, J. (2018). The Design of Visual Programming Languages for Interactive Graphics Applications. ACM Computing Surveys, 50(1), 1-34.
\bibitem{hashemi2017} Hashemi, A., \& Kiani, M. (2017). A computational representation of traditional textile patterns for digital fabrication. International Journal of Design \& Nature and Ecodynamics, 12(4), 438-445.
\end{thebibliography}

\end{document}